% EVALUATION
% Identification
% - Overhead
%   + Which overhead
%   + How is it measured
%   + Is it being noticed
% - How long does it take depending on type of application and how much it is used
%   + comperation of Hostnet applications with graphs
%   + setup of measurement
%   + measurement period
%   + measurement servers
% Elimination
% - Web application visualization (tree map)
%   + is the displayed data usefull
%   + is it usefull in determining which parts to remove
% - Eclipse plugin
%   + does it hinder the developer in daily work
%   + does it prevent mistakes
%   + is it clear what the colors mean to the developers
% Applicability (other organizations)
% Threats to validity





\chapter{Evaluation}
\label{ch:evaluation}

In this chapter we will discuss and evaluate the proposed method for dead code elimination. The evaluation is done by implementing dead code elimination at \furl{hostnet}. Hostnet is a Dutch web hosting company with its own software engineering department of about 10 engineers. The main language used at Hostnet is PHP because many people in the web hosting business are familiar with PHP and use it because it is easy to learn and adopt and facilitates rapid development.

We will evaluate both the data gathering part: dead code identification and the data processing part: dead code elimination. For the identification part we are mainly interested in the overhead of doing runtime analysis and in determining how long the analysis should be performed before we an say with sufficient confidence that code is indeed dead. The overhead is measured in the code identification part that is implemented in Aurora, the biggest application (in number of files) in use within Hostnet. Because the overhead is dependent on the number of files in use in the application should be the worst case within Hostnet. We compare 6 applications developed within Hostnet to see how long it will take until we can tell if the files that are not executed can be regarded as dead files. These application are all different in the way they are used, the number of features and how often they are used.

Both the visualizations and the method for dead code elimination are evaluated. The Eclipse plug-in  is installed for all team members and used by 8 of them currently working on PHP projects. This plug-in is used in the development of all applications done at the moment. The Eclipse plug-in should not hinder the engineers and should we would like to know if it helps the developers by letting them focus on files that are actually used. This is difficult to measure in numbers so we used a questionnaire to evaluate the Eclipse plug-in.

The web application visualization with the tree map is only used by 3 people that really removed code form Aurora. For the evaluation of the proposed process of dead code elimination (see \autoref{ch:elimination}) we tried it on Aurora and managed to remove more then 28\% (2740 files) of the files within just a day work. The evaluation of the visualisation will focus on how the web application is used by the engineers and how they base their decisions on the displayed information (see \autoref{ch:visualization} for a description of the user interface).

For the evaluation of both the visualizations a questionnaire is used. For the evaluation of the dead code elimination process as described in \autoref{ch:elimination}, the evaluation is based on the observations made while the engineers removed the files and monitoring the application to see if there would be any bugs that were caused by the deletion of files. 

The chapter will end with an discussion on the applicability of the method for dead code elimination described in this thesis and the threats to validity of the findings reported in the evaluation.

%Applicability
%Overhead
%Accuracy


\section{Overhead}
\label{sec:overhead}

The overhead caused by reporting all used files to the central database should be low enough to not be noticed by the end user of the application. At the end of each request from the browser for a new page, a list of used files is sent to the central database. Prior to building up the connection and sending the list, the response from the web application is flushed to the browser. This way the browser will already render the web page and only indicates that the page is still loading. This is convenient because even when the overhead is noticeable the end user can still access the application without delay.

We look at the overhead in time per page request. Profiling code has been added to the analysis in Aurora to measure the overhead. Aurora is the most used application within Hostnet and has the biggest number of files. The overhead in Aurora should be the worst case within Hostnet, as the time to send the list to the database and process it, is dependent on the size of the list and the load on the database server. Concurrent write operations to a single MySQL data base on the same row will lock and create bigger load times when the load on the server increases. It is very likely that the same row will be updated in every request to the database because every request to Aurora will pass through the \verb|index.php| file which is the starting point for the application.

When measuring the overhead, the time spent on the analysis is measured as well as the time required to build a connection to the database to be able to see if network congestion could be a limiting factor. A plot of this overhead can be viewed in \autoref{fig:overhead}. On the x-axis the time spent in milliseconds is shown. On the left y-axis the frequency of the corresponding amount of overhead is displayed. In red the overhead for the creation of the connection to the MySQL database on the database server is plotted. The time needed to build the connection has an average at 1.6~ms with and standard deviation of 0.6~ms. The connection overhead seems normally distributed with a standard deviation of under 1~ms and without additional peaks which indicates there is no congestion on the network. In green the frequencies of total overhead have been plotted. The average overhead is 7.8~ms with an standard deviation of 153.5~ms. The blue curve denotes the probability that the overhead will be less than the corresponding value on the x-axis. Here we can see that in over 95\% of the measured cases the overhead stays below 6~ms.

\image[.9\textwidth]{overhead}{Overhead caused by measuring included files (in memory)}{fig:overhead}

When we use a slower method of storage, for example on disk storage in the database server we notice that two peaks will appear in the total overhead curve (see \autoref{fig:overheaddisk}). These appear because if the table is locked the query will have to wait a few milliseconds. Using disk storage, the average overhead is 27.5~ms with an standard deviation of 231~ms. Still in 80\% of the cases the overhead is below 10~ms. In 95\% of the cases the overhead is below 23~ms. The problem of locking rows is clearly visible here. The use of a memory table will not solve this problem but helps us determine where the peaks came from. For the analysis at Hostnet the slower on disk storage is used so that there is no need to build a backup mechanism to make sure that when the database server would be restarted the data would not be lost. 

\image[.9\textwidth]{overhead_disk}{Overhead caused by measuring included files (on disk)}{fig:overheaddisk}

When the load on the database becomes too high other solutions must be found to keep the overhead within bounds so it will not be noticed. One solution could be to use multiple MySQL servers and aggregate the date between them less frequently then the normal updates occur. An other solution could be to aggregate the data on the web server side in shared memory and send it to the server at a frequency less then that of the page requests.

From the experience at Hostnet we can say that the overhead was not noticed by the helpdesk and account managers using Aurora for every day work even when using disk storage. We found that the limiting factor is not the project size but the frequency of requests because locks in the database will occur that create a bigger overhead than the time needed to send the list of files to the database.


\clearpage
\section{Dead Code Identification}

In this section we look at how long we have to wait until we can use the results of the dynamic analysis. Therefore a case study has been done. The overhead measurements in the previous section are gathered from this use case. For about three months 6~Hostnet applications are analysed. The applications differ in age, the number of users that uses them and  the complexity in number of features they possess. In \autoref{tbl:ages} the applications used for the case study are shown. The number of users and features are relative to the other applications. We present the idea that small applications with a lot of unique users will offer certain results faster then a big application with little features. Further we check if the older applications within Hostnet have more dead code than the recently developed ones.

For every application a short introduction about the use of the application in the organization will be given together with a graph that visualizes the number of files used by the application over time. When this graph stabilizes it means that the probability that new files will be included decreases. The graph shows the number of files in the application accessed over time. The last data point in the graph denotes the end of the measurement and not a new file that is included.

All Hostnet applications are build as web applications, written in PHP. All tested applications make use of the \furl{symfony} framework and are connected to a \furl{mysql} Relational Database Cluster. Using Symfony also implies using the auto load functionality the framework offers. This makes all applications suitable for this analysis. The use of MySQL or Symfony is not needed for the implemented identification method to work, only the dynamic loading of files is.

\subsection*{Applications in the use case}
In \autoref{tbl:ages} the 6~applications that will be used in the use case are shown. The applications are developed in the past 6 years and all written in PHP. There are 4~applications, Aurora, My Hostnet, the Web shop and Mailbase that are accessed by humans, the other two, HFT2 and HFT3 are provisioning system that run (mostly) without user interaction. From the 4~web applications Aurora and Mailbase are only for internal use within Hostnet. The Web shop and My Hostnet are accessed by customers. The value of page requests for Aurora in the table has been adjusted to take the number of automated request from some monitoring systems into account. These systems are responsible for almost half of the requests but always request the same page. An example is the TV which shows real-time data about which customers are currently in the web shop.

\begin{table}[b]
	\small
	\centering
	\begin{tabular}{|l|l|l|l|c|r|}
	\hline
	Name & Description & \# files & Age & \# page requests / day\\
	\hline
	HFT3        & New provision system        & \HftThreestrippedFileCount & 1 year  & n.a.    \\
	Web shop    & web shop                    & \OntrackFileCount          & 3 years & 39,643  \\
	Aurora      & CRM application             & \AuroraFileCount           & 5 years & 60,383  \\
	Mailbase    & Lagacy mail filter frontend & \MailbaseFileCount         & 5 years & 2,803   \\
	My Hostnet  & Customer portal             & \MyTwoFileCount            & 5 years & 53,495  \\
	HFT2        & Provisioning system         & \HftTwoFileCount           & 6 years & n.a.    \\
	\hline
	\end{tabular}
	\caption{Applications ordered according to age	\label{tbl:ages}}
\end{table}



%\addtocounter{footnote}{-1}
%\footnotetext{HFT3 included a plugin for the \orm which is available through the Symfony framework for the other applications, this plugin is not taken into account here.}
%\addtocounter{footnote}{1}

\newcommand{\hftTwoFootnote}{HFT2 uses static includes in the legacy part of the application which is not supported. Static included files always report as alive, increasing the percentage of used files significantly. Not all files are included statically so the method is still usable to identify dead code and eliminate it but comparison to the other applications does not make much sense.
}
%\footnotetext{\hftTwoFootnote}



\subsection*{Aurora}
Aurora is the biggest (in files) and most feature rich application developed at Hostnet. It is used to store all customer information. Everything from contact information, bought products, contracts and invoices. Aurora takes care of a lot of the periodic tasks like creating invoices\cite{boomsma2008}. Aurora also offers all kind of statistics and development tools for the Software Engineering department. Aurora is used for daily business by the majority of employees, most work consists of looking up customer data, like contracts, contact data and products. All the advanced function are only used once in a while.

The server has to take care of about 4 visits per second on average. This is because all helpdesk employees use Aurora for almost all their tasks and some TVs which show real-time statistics via Aurora.

Aurora is build up in clearly distinctive modules, batch jobs, database models, plug-ins and templates.
When looking at the. All modules that are not changed recently are probably dead. Before removing a module it is needed to know what it is used for and if it was already expected to be used or not. If it was not expected to be used it can be removed. This shows that selecting files for removal requires human reasoning.

\bottomimage[1\textwidth]{usecase/aurora_saturation}{Number of used files over time in Aurora}{fig:aurora_saturation}

When looking at the graph for the application in \autoref{fig:aurora_saturation} we can see it rises quickly and then keeps growing steadily, the slope of the graph never reaches 0. At the end of January an increase in the number of files is visible, here a new version is deployed and a cache rebuild took place. The cache rebuild used a lot of new files in the database model of Aurora. That the graph is increasing unit now means that new features are still being accessed every day even after 3 month time. For Aurora as a whole it is very likely that new files will be used in the future, however sub modules may be stable in less time. 

Because in Aurora as a whole new files are being accessed every day, it is not possible to draw a general conclusion for the whole application other than that it is not yet possible to start with dead file removal other than for clearly separated parts like modules that only contain dead files or core functionality which is frequently since the start of the analysis.

The first moment when the data becomes useful is at the point the slope decreases substantially and the graph bends. In the graph of Aurora this point is visible at the 14th of January. From this point all base functionality has been accessed. Aurora has a lot of monthly jobs, so we should at leas wait a full month until we start using the data. After this time we can start with the dead code removal. The graph should not shows sudden increases in the number of used files any more at this time, if it does there might be something wrong with the estimation of the frequency wherein features are used.

If we would look at the graph and make an educated guess about the asymptote it will approach of about 5000 just to be on the safe side, we could calculate the probability of deleting a file that will be used in the future (when randomly choosing one). 9755 total files - 5000 alive files = 4755 dead files. 4983 files are not used yet. 4984 unused files - 4755 dead files = 229 alive files not accessed yet. 229 dead files not accessed yet / 4983 files not used yet = 0.046. This gives a change of 4.6\% that a file would be removed that was not dead when random selecting a file. These numbers are not included in the visualization because they are based on a educated guess of where the asymptote would be. Building a model to predict the probability of removing alive files at a given point in time is left as future work.

\subsection*{My Hostnet}

My Hostnet is the customer portal of Hostnet. When somebody owns a domain name, virtual private server or hosting they can login to My Hostnet to view their address data, invoices, products and so on. It is also possible to change settings for products, upgrade products, pay invoicess or cancel contracts.
\\
\hereimage[1\textwidth]{usecase/my2_saturation}{Number of used files over time in My Hostnet}{fig:my2_saturation}


My Hostnet has a lot more different users than Aurora and far less features. The graph in \autoref{fig:my2_saturation} shows a quick rise over just several days and then the slope is 0 for more than a month, then a couple of new files is included because a very rarely used feature has been accessed (in this case an error page that was resurrected for use in a new feature). Although new files may always be used in the future, the dead file identification in My Hostnet stabilizes after about 7~days. Then it is reasonable safe to remove the dead files according to the described dead code elimination procedure. We can conclude that we can start measuring after 14 days. All features of My Hostnet are expected to be used at least once a week. With the exception for error pages. The 7~days needed to stabilize is taken form the graph based on the fact that no new files were accessed afterwards for several months. When using the same method as for Aurora to calculate the probability as we did for Aurora the probability of removing an alive file when random selecting one approaches 0.


\subsection*{Web shop}

The web shop has a high number of page requests and a relatively low number of files (see \autoref{tbl:ages}, although the web shop is highly dynamic and can generate different versions of the web shop based on where the customer came from. Because not all features of the web shop will be activated at all times and there are some very specific feature for rarely sold products, it is to be expected that the web shop will stabilize faster than Aurora but slower than My Hostnet.

When looking at the graph in \autoref{fig:ontrack_saturation} we see that after 30 days still some new files where used. But we can also see is increasingly more time between new files being included. We can also see that the graph became stable after half February. Around 20 February the slope of the graph already was 0 but this was only for a few days. The distance between the point should be bigger than the expected frequency of the features of the application. For the web shop we know that some products are sold only once per month. It is possible, however, to start dead code elimination at this point as long as we keep in mind that all files related to products, especially the ones sold less often, could still be needed.
\hereimage[1\textwidth]{usecase/ontrack_saturation}{Number of used files over time in the web shop}{fig:ontrack_saturation}

When looking into the measurements concerning the web shop we can see that a lot of code that is identified as dead code actually belongs to deactivated features in the order module. When new features will be activated they will only show up in the graph if the files were already deployed on the production server at the time the files were first indexed. Once again we see that dead code identification on the scale of files needs human reasoning. Here we can also see that when an increase is seen in the graph it may be worth investigating where it came from, because if it was caused by enabling a feature we can still consider the graph stable. In the future automatic notification of such events could be implemented.

\subsection*{HFT2}
HFT stands for Hostnet Full Throttle and is the provisioning system used to order all domains and deliver them to the customers, the HFT2 also sets up hosting accounts and mails all needed information to the owner. There are a lot of products that can be delivered through the HFT2. 

The HFT2 is the oldest application under inspection and is not yet (and never will be) fully converted into a Symfony application. This explains the low percentage of dead files; the unconverted part of the HFT2 uses static inclusion of the needed resources. This marks all included files as alive, not telling if they are actually in use.


\hereimage[1\textwidth]{usecase/hft2_saturation}{Number of used files over time in HFT2}{fig:hft2_saturation}

As a consequence, the applied method only works well on the Symfony part (116 dead files). This, by no means, render the method useless for the other part. Because the application is fairly aged a lot of old files that are never included can be found in the legacy part (935 dead files). The plots also show that not all inclusions are static in the old part of the application because new files are included over time.

When we look at the graph in \autoref{fig:hft2_saturation} we can see that the slope approaches zero around 28th of January but that still some new files are accessed later on. These files are visible as the crosses in the line. The last cross only denotes the end of the analysis. However if we would look at the Symfony part of the application we can see that no new files were included within a month. The files that were included in the old part of the application later on, are some specific products and the logout feature in the interface. This shows that we have to be carefull when removing code that is related to products that are not regularly sold. The logout function that is used after some time shows that some features are seldom used. For seldom used features it is questionable if we should keep them. It is arguable that a feature that is only accessed once in the three months should be removed because it costs more time than performing the task by hand.

From the employees of Hostnet we know that a part of the HFT2 is transformed into Symfony and a old legacy part is still there. In the visualization the legacy part of the application might be recognizable by the fact that most of the files have not changes for 6 years. The fact that there is a folder called Symfony in which all files are more recently added would confirm this.

\subsection*{HFT3}
HFT3 is the successor of HFT2 and uses a new database model which is put in a shared repository. At this time HFT3 is the major application using the new database model but in the future more applications will be using it. Action should be taken log and accumulate data for shared code in a separate dataset which is used by all applications which make use of that code to see which code is in use. Libraries could also be analysed in the same way. At this moment we have to hard code different locations to store usage data in the dynamic analysis. It could be possible to automatically generate the location bases on  \vcs data in the future.

HFT3 is a lot smaller (in files) that HFT2 but it is used just as much. The work rounds in which HFT3 fetches data and execute it's tasks are  a lot shorter than in the previous version. 

Allmost all dead files can be tracked down to the database. 807 dead files are located in an \orm plug-in. This plug-in is not included in the other projects because it is bundled with symfony, but the HFT3 uses a newer version so it is included. The percentage of dead files when not taking this plug-in into account drops from  65\% to 40\%. This percentage is more in line with the measured values in the other applications and shows that newer applications probably have less dead code.

\hereimage[1\textwidth]{usecase/hft3_saturation}{Number of used files over time in HFT3}{fig:hft3_saturation}
When we look at the graph for the HFT3 in \autoref{fig:hft3_saturation} we can see that over time new files are still being accessed. When we look further into the HFT we can see that some of the new files belong to an \orm plug-in (which should be outside of the project). The only new file accessed that does not belong to the \orm plug-in contains an Exception class which was used after one month. This leads to the conclusion that the HFT3 has to be measured for about one month to get sufficient certainty.

\subsection*{Mailbase}

Mailbase is is used to view and search all email that is send to and received from customers as well as  mail to domain registrars. Only the user interface is under inspection. Mailbase is almost never used. The code is written using the Symfony framework and is the smallest (in number of files) of the tested applications. 

\hereimage[1\textwidth]{usecase/mailbase_saturation}{Number of used files over time in Mailbase}{fig:mailbase_saturation}

Because Mailbase is used very little it is hard to know when new features will stop showing up. In the graph in \autoref{fig:mailbase_saturation} we can see that the distance between new files being included does not lengthen yet, so there is no indication to assume the graph is stabilizing. This means we can not use the data for maintenance purposes as we can for the other applications.

At Hostnet only one person is still using Mailbase. Everything that he does not use will be removed from Mailbase and the functionality that remains will be ported to Aurora. In this case the only way to get accurate data is to ask the one person sill using the application. An option is to just port only the functionality used in the past 3 month and add all other functionality only on request with the old Mailbase code in the \vcs as reference. This shows that depending on the goal it is possible to use the dead code identification method even when an application is used by only one user.


\subsection*{Results of the case study}

Now we got data for all applications we can compare them and see if the applications behave as we thought within Hostnet. First we look at the age of the applications. We expected older applications to have a lower percentage of used code than the more recent ones. 

If we order the applications according to their age, and we compare that with the ranking according ranking according to the percentage of unused files, then we observe that

 \autoref{tbl:used}. In \autoref{fig:all_apps} the graphs that were used to determine when to start with dead code elimination for every application in the use case are combined into one figure. In this figure the x-axis still denotes the time but it is given in days instead of dates, because not all use cases were started at the same time and thus they would start at different dates making it harder to compare them. This is also the reason for the fact that not all lines in the graph end at the same position. The y-axis uses a percentage instead of the number of files so it is possible to compare the applications in this graph according to percentage of used code. This figure shows that, in general, no conclusion can be drawn on how long you have to wait before starting dead code elimination.
 
When looking at \autoref{tbl:used} we see that the HFT2 does contain less unused code than the other applications even though it is the oldest one. This is because a big part of the HFT2 is statically included and denoted alive even if it would not be used at all. The HFT2 also has a newer part of the application which uses dynamic inclusion. This part shown as the HFT2 (Symfony) entry in the table. We also notice that the HFT3 uses less of its code then the web shop does despite of it being younger. This can be explained the fact HFT3 contains relatively much shared and generated code. The shared code can be found in the model of the HFT3 which also contains classes for a new mass mailer system.
 
In general we can say that older applications tend to have more dead code within Hostnet. We also see that code reuse is an important source of dead code in the Hostnet applications. Dead code is found in plug-ins for all application.

The second prediction, that small applications and much used application  will reach a stable analysis faster than little used complex ones. We look at \autoref{tbl:time} to see if this is the case for the applications at Hostnet. The division of the number of files by the number of page views per day is taken to compare it to the time needed to stabilize we found in de case study. We see that the prediction is only partly confirmed. The web shop took longer than was to be expected based solely on the number of features and users. We see that it is important to know what type of features an application posses. For example in the web shop some rarely sold products are available that require extra information from the client. The files needed for those products are expected to be only used once a month. This demonstrates that a rough estimation can be made on the number of files and users but that for a better estimation it is needed to know when the application features are expected to be used. We can also see that for Hostnet the time we had to wait to be certain ranges from 1 week to (at least) over 3 month. It is possible to already eliminate dead code from stable parts of the system. In the use case on dead code elimination (next section) we successfully removed dead files from Aurora where new files are accessed every day.

\begin{table}[p]
	\small
	\centering
	\begin{tabular}{|l|l|l|l|l|c|c|}
	\hline
	Name & Description & Age & \% used\\
	\hline
	HFT3    		         & New provision system         & 1 year  & \HftThreestrippedPctAlive\% \\
	Web shop             & web shop                     & 3 years & \OntrackPctAlive\%          \\
	HFT2 (Symfony)       & Provisioning system          & 3 years & 50.64\%                     \\
	Aurora               & CRM application              & 5 years & \AuroraPctAlive\%           \\
	Mailbase             & Lagacy mail filter frontend  & 5 years & \MailbasePctAlive\%         \\
	My Hostnet           & Customer portal              & 5 years & \MyTwoPctAlive\%            \\
	HFT2\footnotemark    & Provisioning system          & 6 years & \HftTwoPctAlive\%           \\
	\hline
	\end{tabular}
	\caption{Applications ordered according to age with percentage of used files\label{tbl:used}}
\end{table}

\begin{table}[p]
	\small
	\centering
	\begin{tabular}{|l|r|r|r|l|}
	\hline	 
	 Name        & page views $/$ day & \# files & $\text{\# files}/\text{\# page views}$  & no new files accessed after\\
	\hline
	My Hostnet & 53,495 & \MyTwoFileCount            &  0.045 & 1 week\\
	HFT3       & n.a.   & \HftThreestrippedFileCount &   n.a. & 1 month\\
	Web shop   & 39,643 & \OntrackFileCount          &  0.023 & 1 month\\
	HFT2       & n.a.   & \HftTwoFileCount           &   n.a. & 2 months\\
	Aurora     & 60,383 & \AuroraFileCount           & 0.161 & Not yet\\
	Mailbase   &  2,803 & \MailbaseFileCount         & 0.171 & uncertain\\
	\hline 
	\end{tabular}
	\caption{Time to wait before now new files are accessed any more per application\label{tbl:time}}
\end{table}
\footnotetext{\hftTwoFootnote}

\pageimage{usecase/all}{Ratio of used files per application}{fig:all_apps}


% Elimination
% - Web application visualization (tree map)
%   + is the displayed data usefull
%   + is it usefull in determining which parts to remove
% - Eclipse plugin
%   + does it hinder the developer in daily work
%   + does it prevent mistakes
%   + is it clear what the colors mean to the developers

\section{Dead code elimination}



In this section the dead code elimination process using the created visualization tools is evaluated. The process and tree map visualization is used by two people for a full day to test the described process (\autoref{ch:elimination}) and remove dead parts of Aurora. The Eclipse plug-in is installed for all team members and has been used for a full month by the time of writing. Both visualizations and the process are evaluated using a questionnaire, because it is extremely difficult to measure objectively how the visualizations aids the developers eliminating dead code. The Eclipse plug-in could already reduce some of the maintenance cost of dead code without actually removing it by preventing the developers to mistakenly search for a bug or search for a functionality in a file that is not used.

For the visualization containing the tree map, table an graph of used files we look at how the various components of the interface are used and if the user finds them useful. This is done in the questionnaire with the questions that can be seen in \autoref{tbl:treemap}.

\begin{table}
\begin{tabular}{p{7cm}ccccc}
%\toprule
\textbf{Question:} 
& \rotatebox{90}{fully agree}
  \rotatebox{90}{very useful}
& \rotatebox{90}{agree}
  \rotatebox{90}{useful}
& \rotatebox{90}{disagree}
  \rotatebox{90}{useless}
& \rotatebox{90}{completely disagree}
  \rotatebox{90}{completely useless}
& \rotatebox{90}{do not know}
\\

\cmidrule(r){1-1} \cmidrule(lr){2-2} \cmidrule(lr){3-3}\cmidrule(lr){4-4}\cmidrule(lr){5-5}\cmidrule(lr){6-6} 

The tree map helps me to find dead code:                                         & 3 &   &   &   & \\
I know where I am in the project structure when browsing the directory structure & 1 & 1 & 1 &   & \\
I use the graph in my decisions about code removal                               & 1 & 2 &   &   & \\
I use the table when looking for dead code                                       & 2 & 1 &   &   & \\
Rate the usefulness of the following entries in the table                        &   &   &   &   & \\
\hspace{2em} path                     &3& & & &\\
\hspace{2em} percentage of dead files & &3& & &\\
\hspace{2em} number of dead files     &1&1&1& &\\
\hspace{2em} number of .php files     &1&1&1& &\\
\hspace{2em} number of hits           &1&1&1& &\\
\hspace{2em} age                      &3& & & &\\
\hspace{2em} hit time                 &1&2& & &\\
%\bottomrule
\end{tabular}

\caption{Questions about the tree map visualization\label{tbl:treemap}}
\end{table}

As is visible in \autoref{tbl:treemap}, the questionnaire was only taken by 3 people because it was not possible to spare more people for a day of dead code elimination. Two of them worked one full day on removing dead code from Aurora using the tree map visualization. When we look at the answers we can see that the tree map points users in the right direction to find the unused files that can be removed and that it is found helpful. The same is true for the table and the graph. When looking at the columns in the table we can see that data that is already available in the tree map is rated less useful then data that is available only in the table. Also the total number of PHP files seems of less interest than the other data. Nobody rated a part of the interface completely useless. Maybe the interface should be improved to make it more visible which project and directory is currently viewed because this was not clear to all participants. Although it is not possible to draw general conclusions from such a small data set, it is possible to say that the visualization did help the engineers to locate, select end eliminate dead code. 

We managed to reduce the size of Aurora by 28\% in less then a day's work with two people. So far, no defects were found related to the  removal of that dead code.

The Eclipse plug-in was also evaluated using the same questionnaire. For the Eclipse plug-in 8 people gave their opinion on how they use and like the plug-in. The participants consist of three senior developers, five junior developers. The questions and aggregated answers are in \autoref{tbl:eclipse}.


\begin{table}
\begin{tabular}{p{9cm}ccccc}
%\toprule
\textbf{Question:} 
& \rotatebox{90}{fully agree}
& \rotatebox{90}{agree}
& \rotatebox{90}{disagree}
& \rotatebox{90}{completely disagree}
& \rotatebox{90}{do not know}
\\

\cmidrule(r){1-1} \cmidrule(lr){2-2} \cmidrule(lr){3-3}\cmidrule(lr){4-4}\cmidrule(lr){5-5}\cmidrule(lr){6-6} 

I have the dead files plug-in enabled all the time            & 5 & 3 &   &   & \\
I turn the plug-in only on for the purpose of removing files  &   &   & 4 & 4 & \\
The plug-in aids me when debugging applications               & 4 & 2 & 2 &   & \\
The plug-in aids me when developing new features              &   & 5 & 3 &   & \\
A file that is red can be removed                             &   & 1 & 4 & 3 & \\
A fully green folder only contains live .php data             & 4 & 4 &   &   & \\    

%\bottomrule
\end{tabular}
\caption{Questions about the Eclipse plug-in\label{tbl:eclipse}}
\end{table}
When we look at the answers (see \autoref{tbl:eclipse}), we observe that people have the plug-in turned on all the time and do not turn it off despite the fact that they were shown how that would be done. This leads to the conclusion that the plug-in does not get in the way of the developers. Next we want to know if the plug-in also aids the developers with their everyday work and if they understand what de colors mean. Most people find it especially useful when debugging, some others also when developing new features. When looking at the raw data it is possible to see that everybody found the plug-in useful for at least debugging or developing new features. The last to questions were added to test whether the participants knew what the colors precisely mean. A file that is red is not used in production, but on only that fact is not enough to justify removal. A fully green folder does indeed only contain used .php files, although it is possible that it will contain other unused resources. We can conclude that in general people do know what the colors mean and find them useful in their everyday work. I got a lot of positive feedback about the plug-in from the team at Hostnet.

\section{Discussion}
In this section the findings and short coming of the presented method for dead code identification and elimination will be discussed based on the case study results. Also applicability and threads to validity will be looked upon.

The implemented method should be enhanced to be able to measure a shared code base, this will not be much trouble. To also take statically included classes into account more work research should be performed if this is possible with an PHP plugin or that modifying the source is needed.

Outside PHP the method is applicable to all languages which use an auto loading mechanism to dynamically load classes and allow modification of this mechanism or an other method to list all loaded classes. This means the method is applicable for Java, but not for C and C++ programs.

The overhead observed is about 8~ms on average for the fast in memory storage and about 27~ms on average when using disk storage. Overhead only appears at the end of every page, all content is already send to the client, only the connection is not closed yet because this would end the execution of the PHP script, which results in a slightly longer visible load indicator in the browser but all information will be available to the user without delay.

The use cases provided useful data on how to measure an application. They also showed that how long you have to wait before you get useful data is dependent on how many actions are performed on the system and how these actions are distributed along the features and the number of features. For the web shop, customer portal and provisioning system it is not needed to wait longer than 2 month.

When looking at the tested applications measured we observe that an older application usually has a higher percentage dead code within Hostnet. This also is to be expected because more code remains as a program ages. This can be seen in figure \ref{fig:all_apps} and in table: \ref{tbl:used}. This property can not just be generalized, because it depends a lot on how much effort is put in keeping the application small by the development team\cite{scanniello2011}, but within Hostnet everything is build by the same team which makes it possible to compare the applications.

When removing code, searching for code referring to dead code is a tedious job, but the color decorator plug-in in Eclipse aids with this, because not only the files, but also the search results are decorated. This means that when the programmer searches for method invocations of method in a dead file all files containing those invocations will also be coloured in the search result. Automatically indicating the methods that call methods in dead files should be the next step to easy the programmers job.

\subsection*{Applicability}
% dynamic loaded files
% administrate loaded files

The discussed method for identifying dead files can be used in environments where files are dynamically loaded at run time. As long as files are only loaded when there actually used it is possible to add tracing code to the application or use language features to get a list of all used files. The method will not work in environments where files are statically loaded at run time or are compiled in to an executable.

In practice this method will not work for compiled languages like C and C++ because there will be no files any more when the the application is compiled into an executable. For languages that dynamically loads files like Java does for classes 

The tree map visualization and the Eclipse plug-in can be connected to any data source about dead files. For example determining which images or other resources are used in an web application can be done by parsing the web server access log files and fed into the central database. The visualization tools created will now be able to show information for images without customization.

The procedure for dead code elimination is based on the fact that the analysis is dynamic. It should be equally valid for removing other resources or code with a different granularity as long as the data is dynamically gathered.

The tool box created for the case studies can be reused without modification in every PHP application that uses the auto load functionality instead of statically including all files.

\subsection*{Threads to validity}
% - Application with low number of users
% - Applications with static files
% - Developers ... ?? difficult to foresee.
% - Comparing applications is not possible (??)

There are some important threads to validity of the results. The method will only perform well for applications which are used very much. It is not possible to get accurate data from an application only used a few times a weak by a couple of users. A second thread is that the application should not make use of static file inclusion, this would not make the data less reliable but certainly less usable because all static included files will be marked as alive. The use of already available code coverage tools could be used to circumvent this problem in situations where an overhead of several times the normal execution time is no problem.

The toolbox and method are especially developed for use within Hostnet. The method has not been tested on applications outside of Hostnet.

When a comparison will be made between different applications it is important to only measure those files that are maintained by the team the measurement is done for because otherwise unused features of plug-ins or a framework will pollute the statistics.

Elimination of dead files sill is a humans job, this implies that somebody with better knowledge of the code will likely be more efficient in removing the dead code than somebody that is not familiar with the application at hand. This could compromise a clear view on how well this method performs and how the tools aid the developer. Somebody that knows the application well can be more accurate for some parts than is possible then when only using the data from the visualizations.

The current overhead data is only valid for a MySQL database. It could be worse when using an alternative way of storage for the central database. This said, MySQL is freely available to everybody and could be used when needed.

The usefulness of the visualization and the elimination of dead code is only evaluated within Hosnet with 8 developers from the software engineering department. It is not possible to draw conclusions from this data set although the results look promising.